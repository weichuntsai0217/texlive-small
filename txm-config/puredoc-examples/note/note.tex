\documentclass[10pt, a4paper, oneside]{article}
\usepackage[utf8]{inputenc}

% To assign English fonts
% \usepackage{fontspec}
% \setmainfont{Roboto-Regular.ttf}[
%   % Path=/Some Folder Your Fonts/ , % If you want to assign folder path which contains your font files.
%   BoldFont=Roboto-Bold.ttf ,
%   ItalicFont=Roboto-Italic.ttf ,
%   BoldItalicFont=Roboto-BoldItalic.ttf
% ]

% To assign Chinese (CJK) fonts
% \usepackage[AutoFakeBold=1, AutoFakeSlant=0.2]{xeCJK}
% \setCJKmainfont{NotoSerifTC-Regular.otf}[
%   BoldFont=NotoSerifTC-Bold.otf
% ]

\setlength\parindent{0pt}


\usepackage{geometry}
\geometry{
  left=3.0cm,
  top=2.2cm,
  right=3.0cm,
  bottom=2.5cm,
  footskip=1.0cm
}


\usepackage[colorlinks=true, allcolors=blue, unicode]{hyperref}
\usepackage{enumitem}
\providecommand{\tightlist}{%
  \setlength{\itemsep}{0pt}\setlength{\parskip}{0pt}}

\usepackage{longtable,booktabs}

\usepackage{graphicx}

\usepackage{imakeidx}
\makeindex[intoc]

\pagestyle{empty}




\begin{document}




\section{What is puredoc?}\label{what-is-puredoc}

\textbf{puredoc} is a docker image for document generation based on
\href{https://github.com/weichuntsai0217/texlive-small}{texlive-small} and
\href{https://pandoc.org/}{pandoc}.
It allows you to generate documents in multiple file formats
from single yaml/markdown source.

\section{Installation}\label{installation}

\begin{itemize}
\item
  Step 1. Install \texttt{docker} by following \href{https://docs.docker.com/install/}{Docker official installation guide}.
  Docker is cross-platform so you can run \textbf{puredoc} on Mac OSX / Linux / Windows.
\item
  Step 2. To download the image, run

\begin{verbatim}
$ docker pull weichuntsai/puredoc:1.0
\end{verbatim}
\item
  Step 3. To initialize the container, run

\begin{verbatim}
$ docker run --name ${container_name} -dt \
  -v ${host_dir}:${container_dir} weichuntsai/puredoc:1.0
\end{verbatim}

  For example,

\begin{verbatim}
$ docker run --name mydoc -dt \
  -v /Users/jimmy_tsai:/home weichuntsai/puredoc:1.0
\end{verbatim}
\item
  Step 4. To enter into the container you initialized in Step 3 (assume its name is \texttt{mydoc}), run

\begin{verbatim}
$ docker exec -it mydoc bash
\end{verbatim}

  If you initialize the container as the example of Step 3,
  then you would find that all contents in the \texttt{/Users/jimmy\_tsai} of the host computer
  would show up in the \texttt{/home} of the container.
\end{itemize}

For what commands you can use in this container, please refer to the following \textbf{Usage} parts.

\section{Usage - README}\label{usage---readme}

To see \texttt{README} in the container, just run:

\begin{verbatim}
$ more /README-puredoc.md
\end{verbatim}

\section{Usage - generate documents}\label{usage---generate-documents}

The command is as below

\begin{verbatim}
$ puredoc ${my_input_file} ${template_type} ${output_extension} \
  -o ${output_name_without_ext} -c "${CJK_font_name}" -d ${output_dir}
\end{verbatim}

where \texttt{-o} (\texttt{-\/-output}), \texttt{-c} (\texttt{-\/-cjk}), \texttt{-d} (\texttt{-\/-dir}) are options you don't necessarily have to provide them.

The \texttt{template\_type} supports \texttt{resume}, \texttt{coverletter}, \texttt{article}, or \texttt{book}
and the \texttt{output\_extension} depends on the \texttt{template\_type} you choose and would be explained in the following sections.

\subsection{To generate a resume / CV}\label{to-generate-a-resume-cv}

\begin{verbatim}
$ puredoc ${my_input_file} resume ${output_extension}
\end{verbatim}

For the template type \texttt{resume},
the input file should be a \texttt{yaml} and
the allowed output extensions are \texttt{pdf}, \texttt{md}, \texttt{txt} and \texttt{tex}.
For example:

\begin{itemize}
\item
  To generate \texttt{my-resume.pdf}

\begin{verbatim}
$ puredoc my-resume.yaml resume pdf
\end{verbatim}
\item
  To generate \texttt{my-resume.md}

\begin{verbatim}
$ puredoc my-resume.yaml resume md
\end{verbatim}
\item
  To generate \texttt{my-resume.txt}

\begin{verbatim}
$ puredoc my-resume.yaml resume txt
\end{verbatim}
\item
  To generate \texttt{my-resume.tex}

\begin{verbatim}
$ puredoc my-resume.yaml resume tex
\end{verbatim}
\item
  To generate \texttt{my-resume.pdf}, \texttt{my-resume.md}, \texttt{my-resume.txt}, \texttt{my-resume.tex} at one time

\begin{verbatim}
$ puredoc my-resume.yaml resume all
\end{verbatim}
\end{itemize}

If you want to change the output file name, you can use \texttt{-o} or \texttt{-\/-output} with a file name without extension.
For example, you want the output file is \texttt{your\_resume.pdf}, then please run

\begin{verbatim}
$ puredoc my-resume.yaml resume pdf -o your_resume
\end{verbatim}

If you provide \texttt{-o} but not with a file name you want,
the output file would be the form like \texttt{output-of-puredoc-2018-12-24-12-39-03.pdf}

If you want to change the output directory (default is the current directory), you can use \texttt{-d} or \texttt{-\/-dir} with a directory path.
For example, you want the output directory is \texttt{/aa/bb}, then please run

\begin{verbatim}
$ puredoc my-resume.yaml resume pdf -d /aa/bb
\end{verbatim}

If you provide \texttt{-d} but not with a directory path,
the output directory would be \texttt{output/} in the current directory.

If you have non-English content (for example, traditional Chinese), you can use \texttt{-c} or \texttt{-\/-cjk} with a CJK font name.
For example,

\begin{verbatim}
$ puredoc my-resume.yaml resume pdf -c "WenQuanYi Zen Hei" 
\end{verbatim}

If you provide \texttt{-c} but not with a CJK font name,
\texttt{puredoc} would use ``AR PL UMing TW'' (the only pre-installed CJK font in puredoc image) as the default CJK font.

Your input file can also be from the pipe data like this:

\begin{verbatim}
$ cat my-resume.yaml | puredoc resume pdf -o your_resume
\end{verbatim}

If your input file is from pipe data, then you have to use \texttt{-o} option like above to assgin the output file name.

\subsection{To generate a coverletter}\label{to-generate-a-coverletter}

\begin{verbatim}
$ puredoc ${my_input_file} coverletter ${output_extension}
\end{verbatim}

For the template type \texttt{coverletter},
the input file should be a \texttt{yaml} and
the allowed output extensions are \texttt{pdf}, \texttt{md},\texttt{txt} and \texttt{tex}.
About the examples, please refer to \textbf{To generate a resume / CV} above.
The only difference is the template type.

\subsection{To generate an article}\label{to-generate-an-article}

\begin{verbatim}
$ puredoc ${my_input_file} article ${output_extension}
\end{verbatim}

For the template type \texttt{article},
the input file should be a \texttt{md} and
the allowed output extensions are \texttt{pdf} and \texttt{tex}.
About the examples, please refer to \textbf{To generate a resume / CV} above.
The only difference is the template type and the extension of the input file.

\subsection{To generate a book}\label{to-generate-a-book}

\begin{verbatim}
$ puredoc ${my_input_file} book ${output_extension}
\end{verbatim}

For the template type \texttt{book},
the input file should be a \texttt{md} and
the allowed output extensions are \texttt{pdf} and \texttt{tex}.
About the examples, please refer to \textbf{To generate a resume / CV} above.
The only difference is the template type and the extension of the input file.

\subsection{To generate a big book / article}\label{to-generate-a-big-book-article}

Sometimes you want to divide a big \texttt{md} into several \texttt{md} and then compile them into a pdf.
To do this, you have to provide a \texttt{txt} file (assume it is \texttt{my-big-book.txt}) containing your \texttt{md} file names like this

\begin{verbatim}
title-preface-toc-bib.md
chapter-1.md
chapter-2.md
chapter-3.md
\end{verbatim}

Then you can run:

\begin{verbatim}
$ puredoc my-big-book.txt book pdf
\end{verbatim}

For more details, please refer to \texttt{examples/book-with-multiple-files/}

\section{\texorpdfstring{How the input \texttt{yaml} or \texttt{md} look like?}{How the input yaml or md look like?}}\label{how-the-input-yaml-or-md-look-like}

\textbf{Please refer to \texttt{examples/} for more details.}

\section{Special characters in yaml files or yaml header of markdown files}\label{special-characters-in-yaml-files-or-yaml-header-of-markdown-files}

To show \texttt{:} in \texttt{pdf}, \texttt{md} and \texttt{txt} normally,
you have to type one space with \texttt{\textgreater{}} just at the right of the key and
put your value to the next line of the key.

For example, if our key is ``tech'' and our value is ``Frontend technologies: React, Redux'',
then you should type words as below:

\begin{verbatim}
# In your yaml file or yaml header of the markdown file
---
tech: >
  Frontend technologies: React, Redux
...
\end{verbatim}

To show other common special characters \texttt{@\ -\ \%\ \&\ \textbar{}\ +\ \{\ \}\ \#\ \textbackslash{}\ *\ \$\ \^{}\ \_\ \textgreater{}\ \textless{}\ {[}\ {]}}
in \texttt{pdf}, \texttt{md} and \texttt{txt} normally, please refer to the following table:

\begin{longtable}[c]{@{}cc@{}}
\toprule
What to show & What you have to type in \texttt{.yaml}\tabularnewline
\midrule
\endhead
@ & \textbackslash{}@\tabularnewline
- & \textbackslash{}-\tabularnewline
\% & \textbackslash{}\%\tabularnewline
\& & \textbackslash{}\&\tabularnewline
\textbar{} & \textbackslash{}\textbar{}\tabularnewline
+ & \textbackslash{}+\tabularnewline
\{ & \textbackslash{}\{\tabularnewline
\} & \textbackslash{}\}\tabularnewline
\# & \textbackslash{}\#\tabularnewline
\textbackslash{} & \textbackslash{}\textbackslash{}\tabularnewline
* & \textbackslash{}*\tabularnewline
\$ & \textbackslash{}\$\tabularnewline
\^{} & \textbackslash{}\^{}\tabularnewline
\_ & \textbackslash{}\_\tabularnewline
\textgreater{} & \textbackslash{}\textgreater{}\tabularnewline
\textless{} & \textbackslash{}\textless{}\tabularnewline
{[} & \textbackslash{}{[}\tabularnewline
{]} & \textbackslash{}{]}\tabularnewline
\bottomrule
\end{longtable}


\printindex
\end{document}
