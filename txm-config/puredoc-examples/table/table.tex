\documentclass[10pt, a4paper, oneside]{article}
\usepackage[utf8]{inputenc}

% To assign English fonts
% \usepackage{fontspec}
% \setmainfont{Roboto-Regular.ttf}[
%   % Path=/Some Folder Your Fonts/ , % If you want to assign folder path which contains your font files.
%   BoldFont=Roboto-Bold.ttf ,
%   ItalicFont=Roboto-Italic.ttf ,
%   BoldItalicFont=Roboto-BoldItalic.ttf
% ]

% To assign Chinese (CJK) fonts
% \usepackage[AutoFakeBold=1, AutoFakeSlant=0.2]{xeCJK}
% \setCJKmainfont{NotoSerifTC-Regular.otf}[
%   BoldFont=NotoSerifTC-Bold.otf
% ]

\setlength\parindent{0pt}


\usepackage{geometry}
\geometry{
  left=3.0cm,
  top=2.2cm,
  right=3.0cm,
  bottom=2.5cm,
  footskip=1.0cm
}


\usepackage[colorlinks=true, allcolors=blue, unicode]{hyperref}
\usepackage{enumitem}
\providecommand{\tightlist}{%
  \setlength{\itemsep}{0pt}\setlength{\parskip}{0pt}}

\usepackage{longtable,booktabs}

\usepackage{graphicx}

\usepackage{imakeidx}
\makeindex[intoc]

\pagestyle{empty}




\begin{document}




\section*{Table Examples}\label{table-examples}\markboth{Table Examples}{}
\addcontentsline{toc}{section}{Table Examples}

\section{Standard Markdown Table}\label{standard-markdown-table}

A standard markdown table supports single line only in each column and its width can not be modified.
This table can also be rendered well on common markdown platforms (ex: GitHub).

\begin{longtable}[c]{@{}rllc@{}}
\caption{A Standard Markdown Table}\tabularnewline
\toprule
Right & Left & Default & Center\tabularnewline
\midrule
\endfirsthead
\toprule
Right & Left & Default & Center\tabularnewline
\midrule
\endhead
12 & 12 & 12 & 12\tabularnewline
123 & 123 & 123 & 123\tabularnewline
1 & 1 & 1 & 1\tabularnewline
\bottomrule
\end{longtable}

\section{Extended Markdown Table}\label{extended-markdown-table}

An extended markdown table supports multiple lines in each column.
This table can not be rendered well on common markdown platforms (ex: GitHub).

\begin{longtable}[c]{@{}clr@{}}
\caption{An Extended Markdown Table}\tabularnewline
\toprule
\begin{minipage}[b]{0.14\columnwidth}\centering\strut
Centered
Header
\strut\end{minipage} & \begin{minipage}[b]{0.11\columnwidth}\raggedright\strut
Default
Aligned
\strut\end{minipage} & \begin{minipage}[b]{0.11\columnwidth}\raggedleft\strut
Right
Aligned
\strut\end{minipage}\tabularnewline
\midrule
\endfirsthead
\toprule
\begin{minipage}[b]{0.14\columnwidth}\centering\strut
Centered
Header
\strut\end{minipage} & \begin{minipage}[b]{0.11\columnwidth}\raggedright\strut
Default
Aligned
\strut\end{minipage} & \begin{minipage}[b]{0.11\columnwidth}\raggedleft\strut
Right
Aligned
\strut\end{minipage}\tabularnewline
\midrule
\endhead
\begin{minipage}[t]{0.14\columnwidth}\centering\strut
First
\strut\end{minipage} & \begin{minipage}[t]{0.11\columnwidth}\raggedright\strut
row,
long
content
\strut\end{minipage} & \begin{minipage}[t]{0.11\columnwidth}\raggedleft\strut
12.0
\strut\end{minipage}\tabularnewline
\begin{minipage}[t]{0.14\columnwidth}\centering\strut
Second
\strut\end{minipage} & \begin{minipage}[t]{0.11\columnwidth}\raggedright\strut
row
\strut\end{minipage} & \begin{minipage}[t]{0.11\columnwidth}\raggedleft\strut
5.0
\strut\end{minipage}\tabularnewline
\bottomrule
\end{longtable}

\section{How to Make an Extended Markdown Table Wider?}\label{how-to-make-an-extended-markdown-table-wider}

For an extended markdown table, you can also modify its width by adding more hyphens ``-''.

\begin{longtable}[c]{@{}clr@{}}
\caption{A Wider Extended Markdown Table}\tabularnewline
\toprule
\begin{minipage}[b]{0.27\columnwidth}\centering\strut
Centered
Header
\strut\end{minipage} & \begin{minipage}[b]{0.18\columnwidth}\raggedright\strut
Default
Aligned
\strut\end{minipage} & \begin{minipage}[b]{0.18\columnwidth}\raggedleft\strut
Right
Aligned
\strut\end{minipage}\tabularnewline
\midrule
\endfirsthead
\toprule
\begin{minipage}[b]{0.27\columnwidth}\centering\strut
Centered
Header
\strut\end{minipage} & \begin{minipage}[b]{0.18\columnwidth}\raggedright\strut
Default
Aligned
\strut\end{minipage} & \begin{minipage}[b]{0.18\columnwidth}\raggedleft\strut
Right
Aligned
\strut\end{minipage}\tabularnewline
\midrule
\endhead
\begin{minipage}[t]{0.27\columnwidth}\centering\strut
First
\strut\end{minipage} & \begin{minipage}[t]{0.18\columnwidth}\raggedright\strut
row,
long content,
long content,
long content,
long content,
long content,
\strut\end{minipage} & \begin{minipage}[t]{0.18\columnwidth}\raggedleft\strut
12.0
\strut\end{minipage}\tabularnewline
\begin{minipage}[t]{0.27\columnwidth}\centering\strut
Second
\strut\end{minipage} & \begin{minipage}[t]{0.18\columnwidth}\raggedright\strut
row
\strut\end{minipage} & \begin{minipage}[t]{0.18\columnwidth}\raggedleft\strut
5.0
\strut\end{minipage}\tabularnewline
\bottomrule
\end{longtable}


\printindex
\end{document}
